%----------------------------------------------------------------------------------------
%	PACKAGES AND OTHER DOCUMENT CONFIGURATIONS
%----------------------------------------------------------------------------------------
\documentclass[12pt]{article}
\usepackage[protrusion=true,expansion=true]{microtype} % Better typography
\usepackage{graphicx} % Required for including pictures
\usepackage{wrapfig} % Allows in-line images
\usepackage[utf8]{inputenc}
\usepackage{mathpazo} % Use the Palatino font
\usepackage[T1]{fontenc} % Required for accented characters
\usepackage{verbatim} % comentarios
\linespread{1.05} % Change line spacing here, Palatino benefits from a slight increase by default
\usepackage{multirow}
\usepackage{subcaption}
\usepackage{footnote}
\usepackage{listings}
\usepackage{url}





\makeatletter
\renewcommand\@biblabel[1]{\textbf{#1.}} % Change the square brackets for each bibliography item from '[1]' to '1.'
\renewcommand{\@listI}{\itemsep=0pt} % Reduce the space between items in the itemize and enumerate environments and the bibliography

\renewcommand{\maketitle}{ % Customize the title - do not edit title and author name here, see the TITLE block below
\begin{flushright} % Right align
{\LARGE\@title} % Increase the font size of the title

\vspace{50pt} % Some vertical space between the title and author name

{\large\@author} % Author name
\\\@date % Date

\vspace{40pt} % Some vertical space between the author block and abstract
\end{flushright}
}


\begin{document}

\begin{titlepage}

\newcommand{\HRule}{\rule{\linewidth}{0.5mm}} % Defines a new command for the horizontal lines, change thickness here

\begin{center}
 
%----------------------------------------------------------------------------------------
%	HEADING SECTIONS
%----------------------------------------------------------------------------------------

\textsc{\LARGE Benemerita Universidad Autonoma de Puebla}\\[1.2cm] % Name of your university/college
\textsc{\Large PROYECTO FINAL}\\[0.8cm] % Major heading such as course name
\textsc{\Large Sistemas Operativos II}\\[0.5cm] % Major heading such as course name
\textsc \emph{\large Profesor: Luis Enrique Colmenares Guillen}\\[0.3cm] % Minor heading such as course title

%----------------------------------------------------------------------------------------
%	TITLE SECTION
%----------------------------------------------------------------------------------------

\HRule \\[0.4cm]
{ \huge \bfseries Multiplicacion de matrices}\\[0.4cm] % Title of your document
\HRule \\[1cm]
%----------------------------------------------------------------------------------------
%	AUTHOR SECTION
%----------------------------------------------------------------------------------------
\textsc{\Large Autores}\\[0.3cm] % Major heading such as course name
\begin{minipage}{0.5\textwidth}
\begin{flushleft} \large
\centering \textsc{Placido Velazco Cesar Eduardo\\   201214687} % Your name
\end{flushleft}
\end{minipage}
~
\begin{minipage}{0.3\textwidth}
\begin{flushright} \large
\centering \textsc{Vazquez Martinez Josue\\201245534} % Supervisor's Name
\end{flushright}
\end{minipage}\\ [1cm]

\date{\today} % Date

%----------------------------------------------------------------------------------------
%	LOGO SECTION
%----------------------------------------------------------------------------------------

\includegraphics[width=5cm,height=5cm,keepaspectratio]{buap.png}\newpage% Include a department/university logo - this will require the graphicx package

%----------------------------------------------------------------------------------------
\end{center}

\title{\centering Reporte Multiplicacion de matrices}


\author{\textsc{Reporte} % Author
\\{\textit{Benemerita Universidad Autonoma de Puebla}}} % Institution

\date{\today} % Date

%----------------------------------------------------------------------------------------


\maketitle % Print the title section

%----------------------------------------------------------------------------------------
%	ABSTRACT AND KEYWORDS
%----------------------------------------------------------------------------------------

%\renewcommand{\abstractname}{Summary} % Uncomment to change the name of the abstract to something else

\begin{abstract}
En este documento exponemos cada uno de los elementos que hicieron posible la realizacion de el proyecto final de la materia de sistemas operativos II.Ejemplificamos graficamente cada uno de los posibles valores para las matrices sugueridos por el profesor de la materia.
\end{abstract}


\hspace*{3,6mm}\textit{Keywords:} OpenMPI ,Paralelo ,Matriz,C% Keywords

\vspace{30pt} % Some vertical space between the abstract and first section

%----------------------------------------------------------------------------------------
%	ESSAY BODY
%----------------------------------------------------------------------------------------

\section*{Descripcion de hardware}
\begin{table}[htb]
\centering
\begin{tabular}{| p{4.2cm}| p{10.2cm} |}
\hline
\multicolumn{2}{|c|}{Alienware 14} \\
\hline
Procesador & Intel Core i7-4700MQ, 4 núcleos (8 hilos) a 2.40 GHz. \\
\hline \hline
Memoria & 8 GB \\ \hline
Disco duro & 1 TB SATA \\ \hline
Sistema operativo & Ubuntu 16.04 \\ \hline
\end{tabular}
\label{tabla:anchofijo}
\end{table}
\clearpage
%------------------------------------------------
\section*{Implementación algoritmo secuencial\\ \footnote{Los valores que se corren en el algoritmo son de tipo real, para la prueba con valores de tipo flotante solo basta definir el tipo de valores dentro de el código como float.}}
\lstset{language=C, breaklines=true, basicstyle=\footnotesize}
\begin{lstlisting}[frame=single]
//Creando matrices
int 
  A[n][n], 
  B[n][n], 
  C[n][n];

void fill_matrix(int m[n][n])
{
  int i, j;
  for (i=0; i<n; i++){
    for (j=0; j<n; j++){
      m[i][j] = rand()%(2000 + 1 - 1000)+1000; //Llenamos con valores pseudoaleatorios entre 1000 y 2000
    }
  }
}

int main(int argc, char *argv[])
{
  int l,P;
  double average = 0.0;
  MPI_Status status;  
  MPI_Init (&argc, &argv);
  for(l = 0; l < 100; l++){ //Iteramos 100 veces
  
    clock_t start = clock();
    int myrank, from, to, i, j, k;
    MPI_Comm_rank(MPI_COMM_WORLD, &myrank);
    MPI_Comm_size(MPI_COMM_WORLD, &P);


    if (n%P!=0) {
      if (myrank==0) 
        printf("El tamano de la matriz no es divisible entre el numero de procesadores\n");
      MPI_Finalize();
      exit(-1);
    }

    from = myrank * n/P;
    to = (myrank+1) * n/P;


    if (myrank==0) {
      fill_matrix(A);
      fill_matrix(B);
    }

    MPI_Bcast (B, n*n, MPI_INT, 0, MPI_COMM_WORLD);
    MPI_Scatter (A, n*n/P, MPI_INT, A[from], n*n/P, MPI_INT, 0, MPI_COMM_WORLD);

    //printf("Resolviendo particion %d (desde el elemento %d hasta el %d)\n", myrank, from, to-1);
    for (i=from; i<to; i++){
      for (j=0; j<n; j++) {
        for (k=0; k<n; k++){
          C[i][j] += A[i][k]*B[k][j];
        }
      }
    } 
    MPI_Gather (C[from], n*n/P, MPI_INT, C, n*n/P, MPI_INT, 0, MPI_COMM_WORLD);


\end{lstlisting}
\clearpage
\section*{Implementación algoritmo de Cannon}
\lstset{language=C, breaklines=true, basicstyle=\footnotesize}
\begin{lstlisting}[frame=single]
//Creando matrices
int 
  A[n][n], 
  B[n][n], 
  C[n][n];

void fill_matrix(int m[n][n])
{
  int i, j;
  for (i=0; i<n; i++){
    for (j=0; j<n; j++){
      m[i][j] = rand()%(2000 + 1 - 1000)+1000; //Llenamos con valores pseudoaleatorios entre 1000 y 2000
    }
  }
}

int main(int argc, char *argv[])
{
char *fn = argc > 1 ? argv[1] : "gnuplot.dat";

    int pid, statusz;
    float f[MAXL] = {0.0};
    char fnbase[MAXC] = "", fnplt[MAXC] = "";
    size_t iz;
    FILE *fp = NULL;



  int l,P;
  double average = 0.0;
  MPI_Status status;  
  MPI_Init (&argc, &argv);
  for(l = 0; l < 100; l++){ //Iteramos 100 veces
  
    clock_t start = clock();
    int myrank, from, to, i, j, k;
    MPI_Comm_rank(MPI_COMM_WORLD, &myrank);
    MPI_Comm_size(MPI_COMM_WORLD, &P);


    if (n%P!=0) {
      if (myrank==0) 
        printf("El tamano de la matriz no es divisible entre el numero de procesadores\n");
      MPI_Finalize();
      exit(-1);
    }

    from = myrank * n/P;
    to = (myrank+1) * n/P;


    if (myrank==0) {
      fill_matrix(A);
      fill_matrix(B);
    }

    MPI_Bcast (B, n*n, MPI_INT, 0, MPI_COMM_WORLD);
    MPI_Scatter (A, n*n/P, MPI_INT, A[from], n*n/P, MPI_INT, 0, MPI_COMM_WORLD);


    double kk;
    int vecino_derecho, vecino_izquierdo, vecino_arriba, vecino_abajo;

    int PP = P;
    int x;
    double np = P;
    kk = sqrt(np);
    k = (int)kk;
    if (myrank < k) // below neighbour set
    {
        vecino_izquierdo = (myrank + k - 1) % k;
        vecino_derecho = (myrank + k + 1) % k;
        vecino_arriba = ((k - 1)*k) + myrank;
    }
    if (myrank == k)
    {
        vecino_izquierdo = ((myrank + k - 1) % k) + k;
        vecino_derecho = ((myrank + k + 1) % k) + k;
        vecino_arriba = myrank - k;
    }
    if (myrank > k)
    {
        x = myrank / k;
        vecino_izquierdo = ((myrank + k - 1) % k) + x * k;
        vecino_derecho = ((myrank + k + 1) % k) + x * k;
        vecino_arriba = myrank - k;
    }
    if (myrank == 0 || (myrank / k) < (k - 1))
    {
        vecino_abajo = myrank + k;
    }
    if ((myrank / k) == (k - 1))
    {
        vecino_abajo = myrank - ((k - 1)*k);
    }
    x = 0;

    for(int kk = 0; kk < PP; kk++) //algorithm
    {
        for (i = 0; i < n / PP; i++)
        {
            for (j = 0; j < n / PP; j++)
            {
                for (k = 0; k < n / PP; k++)
                {
                    C[i][j] += A[i][k] * B[k][j];
                }
            }
        }
      }

    MPI_Gather (C[from], n*n/P, MPI_INT, C, n*n/P, MPI_INT, 0, MPI_COMM_WORLD);



\end{lstlisting}
\clearpage

\section*{Marco comparativo}
\begin{table}[!htb]
\centering
  \begin{tabular}{|l|l|l|l|l|l|l|}
    \hline
    \multirow{2}{*}{Dimensiones} &
    \multirow{2}{*}{Cores} &
      \multicolumn{2}{c|}{Tiempo} &
      \multicolumn{2}{c|}{Memoria}\\
     
    &  &Secuencial & Cannon & Secuencial & Cannon\\
    \hline
    16X16 & 1\ & 0.000060\ & 0.000056\ & 7744 kb\ & 7654 kb\ \\
    \hline
    32X32 & 1\ & 0.000270\ & 0.000359\ & 7724 kb\ & \ \\
    \hline
    64X64 & 1\ & 0.002067\ & 0.002784\ & 9856 kb\ & 7643 kb\ \\
    \hline
    128X128 & 1\ & 0.014675\ & 0.021790\ & 10172 kb\ & 9235 kb\ \\
    \hline
    256X256 & 1\ & 0.137754\ & 0.170546\ & 8664 kb\ & \ 7264 kb\\
    \hline
    512X512 & 1\ & 1.395750\ & 1.513075\ & 10980 kb\ & 9236 kb\ \\
    \hline
    1024X1024 & 1\ & 15.256436\ & 11.147821\ & 12972 kb\ & 10346 kb\ \\
    \hline
    2048X2048 & 1\ & 125.764687\ & 93.342564\ &23954682 kb \ & 2295343 kb\ \\
    \hline
  \end{tabular}
  \hfill
    \caption{Comparacion de algoritmos con valores de tipo entero con 1 core.}
\end{table}

\begin{table}[!htb]
\centering
  \begin{tabular}{|l|l|l|l|l|l|l|}
    \hline
    \multirow{2}{*}{Dimensiones} &
    \multirow{2}{*}{Cores} &
      \multicolumn{2}{c|}{Tiempo} &
      \multicolumn{2}{c|}{Memoria}\\
     
    &  &Secuencial & Cannon & Secuencial & Cannon\\
    \hline
    16X16 & 2\ & 0.000044\ & 0.000046\ & 8572 kb\ & 7457 kb\ \\
    \hline
    32X32 & 2\ & 0.000619\ & 0.000148\ & 8576 kb\ & 8554 kb\ \\
    \hline
    64X64 & 2\ & 0.001869\ & 0.000959\ & 9024 kb\ & 8572 kb\ \\
    \hline
    128X128 & 2\ & 0.012188\ & 0.000687\ & 8764 kb\ & 9032 kb\ \\
    \hline
    256X256 & 2\ & 0.154470\ & 0.045504\ & 9680 kb\ & 9420 kb\ \\
    \hline
    512X512 & 2\ & 2.468797\ & 394753\ & 11780 kb\ & 10586 kb\ \\
    \hline
    1024X1024 & 2\ & 4.542345\ & 2.622002\ & 18456 kb\ & 43632 kb\ \\
    \hline
    2048X2048 & 2\ & 135.349123\ & 72.707819\ & 3253465 kb\ & 58088 kb\ \\
    \hline
  \end{tabular}
  \hfill
    \caption{Comparacion de algoritmos con valores de tipo entero con 2 cores.}
\end{table}
\clearpage

\begin{table}[!htb]
\centering
  \begin{tabular}{|l|l|l|l|l|l|l|}
    \hline
    \multirow{2}{*}{Dimensiones} &
    \multirow{2}{*}{Cores} &
      \multicolumn{2}{c|}{Tiempo} &
      \multicolumn{2}{c|}{Memoria}\\
     
    &  &Secuencial & Cannon & Secuencial & Cannon\\
    \hline
    16X16 & 4\ & 0.000039\ & 0.000046\ & 10624 kb\ & 10556 kb\ \\
    \hline
    32X32 & 4\ & 0.000182\ & 0.000148\ & 8536 kb\ & 9112 kb\ \\
    \hline
    64X64 & 4\ & 0.001075\ & 0.000959\ & 10992 kb\ & 8708 kb\ \\
    \hline
    128X128 & 4\ & 0.006956\ & 0.000687\ & 8872 kb\ & 9036 kb\ \\
    \hline
    256X256 & 4\ & 0.057274\ & 0.045504\ & 9149 kb\ & 9304 kb\ \\
    \hline
    512X512 & 4\ & 0.442991\ & 2.94753\ & 10460 kb\ & 10500 kb\ \\
    \hline
    1024X1024 & 4\ & 12.348743\ & 6.622002\ & 324574 kb\ & 6543467 kb\ \\
    \hline
    2048X2048 & 4\ & 89.982345\ & 74.458734\ & 3457528 kb\ & 98745834 kb\ \\
    \hline
  \end{tabular}
  \hfill
    \caption{Comparacion de algoritmos con valores de tipo entero con 4 cores.}
\end{table}
\begin{table}[!htb]
\centering
  \begin{tabular}{|l|l|l|l|l|l|l|}
    \hline
    \multirow{2}{*}{Dimensiones} &
    \multirow{2}{*}{Cores} &
      \multicolumn{2}{c|}{Tiempo} &
      \multicolumn{2}{c|}{Memoria}\\
     
    &  &Secuencial & Cannon & Secuencial & Cannon\\
    \hline
    16X16 & 8\ & 0.000044\ & 0.000037\ & 10816 kb\ & 34567 kb\ \\
    \hline
    32X32 & 8\ & 0.000202\ & 0.000135\ & 8796 kb\ & 42456 kb\ \\
    \hline
    64X64 & 8\ & 0.002413\ & 0.00575\ & 9464 kb\ & 34586 kb\ \\
    \hline
    128X128 & 8\ & 0.010143\ & 0.00277\ & 9284 kb\ & 35678 kb\ \\
    \hline
    256X256 & 8\ & 0.070725\ & 0.020139\ & 9804 kb\ & 3356 kb\ \\
    \hline
    512X512 & 8\ & 0.742964\ & 0.090703\ & 10556 kb\ & 35700 kb\ \\
    \hline
    1024X1024 & 8\ & 5.248933\ & 0.477512\ & 14188 kb\ & 35700 kb\ \\
    \hline
    2048X2048 & 8\ & 16.458642\ & 4.119763\ &223456 kb \ & 28796 kb\ \\
    \hline
  \end{tabular}
  \hfill
    \caption{Comparacion de algoritmos con valores de tipo entero con 8 cores.}
\end{table}

Las matrices utilizadas para cada una de las comparaciones fueron las mismas, con valores generados aleatoriamente en un rango de entre 1000-2000 y 1000.000-2000.000 para las tablas de valores enteros y flotantes respectivamente.\\
En ambas tablas observamos la comparacion entre tiempo y uso de memoria, siendo evidente que los tiempos son muchisimo mas costosos para el algoritmo secuencial.

\section*{Conclusion}
El trabajo realizado y presentado de manera comparativa por medio des tablas y la parte de implementacion de cada uno de los algoritmos propuestos por el profesor para la asignatura de sistemas operativos II,nos permitió trabajar con dos algoritmos, todos fueron corridos en una sola maquina portátil, que por sus sobresalientes especificaciones no se presento ningún problema relevante.
A medida que se probaban el algoritmo secuencial pudimos ver que en matrices pequeñas el problema no parecía complicado, pero a medida que las dimensiones de las matrices crecían, la presentación de resultados era aun mas tardía.
Posteriormente pasamos a la implementación de el algoritmo Cannon, y tuvimos algunos problemas al tratar de correr las primeras pruebas, nos dimos cuenta que algunas versiones permitían el uso de ciertos comandos, eso retraso un poco el avance en la comparación de resultados. La version inicial de Openmpi era la versión 1.02 , posteriormente se instalo la versión 2.0.4 hasta finalmente instalar la ultima que es la 3.0. 
Como lenguaje de programación utilizamos C, por sus potentes características y su amplio soporte con la librería.
De manera general y observando las pruebas en los tiempos que arrojaron las pruebas con cada dimensión de matriz dada, podemos darnos cuenta de la importancia que tiene el paralelismo cuando se trata de problemas donde los valores sean bastantes grandes o sean demasiados.
  
\bibliographystyle{unsrt}

\section*{Bibliografia}

@misc{ASH:2013:Online,
author = {Varios Autores},
title = {Archivo Situacionista Hispano},
year = {1999},
howpublished = {url{http://sindominio.net/ash}},
note = {Accedido 01-10-2013}
}








\end{titlepage}
\end{document}